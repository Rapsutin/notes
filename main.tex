\documentclass{article}
\usepackage{ntheorem, amsmath}
\usepackage{graphicx} % Required for inserting images
\usepackage[framemethod=tikz]{mdframed}

\let\vec\mathbf{}

\theoremstyle{break}
\newmdtheoremenv[
  linecolor=red,
  roundcorner=5pt,
]{definition}{Definition}

\theoremstyle{break}
\newmdtheoremenv[
  roundcorner=5pt,
  hidealllines=true,
  leftline=true,
  linecolor=gray
]{theorem}{Theorem}


\theoremstyle{break}
\newmdtheoremenv[
  roundcorner=5pt,
  hidealllines=true,
  leftline=true,
  linecolor=gray
]{corollary}{Corollary}[theorem]
\let\vec\mathbf{}
\let\t\textnormal{}
\DeclareMathOperator{\im}{im}
\DeclareMathOperator{\curl}{curl}
\DeclarePairedDelimiter{\norm}{\lVert}{\rVert}
\DeclarePairedDelimiter{\inner}{\langle}{\rangle}
\DeclarePairedDelimiter{\abs}{\lvert}{\rvert}

\title{Science notes 2}
\author{Juho Lamminmäki }
\date{March 2024}

\begin{document}
\maketitle

\section{Functions}
\begin{definition}[Vector-valued function]
    A real vector-valued function $f: S \subseteq \R^m \rightarrow \R^n$
    is a rule that assigns to each $\vec{p} \in S$ a unique $f(\vec{p}) \in \R^n$.
\end{definition}
\begin{definition}[Coordinate function]
    A vector-valued function $f: S \subseteq \R^m \rightarrow \R^n$ may be expressed
    as
    \begin{equation*}
        f(\vec{p}) = (f_1(\vec{p}), \ldots, f_n(\vec{p}))
    \end{equation*}
    where $f_i: S \subseteq \R^m \rightarrow \R$ are called coordinate functions of $f$.
\end{definition}
\begin{definition}[Graph of a function]
    The graph of a function $f: D \subseteq \R \rightarrow \R^n$ is the subset
    of $\R^{n+1}$ given by $\{(t, f(t)) : t \in D \} \in \R.$
\end{definition}
\subsection{Grad, curl, div}
\begin{definition}[Curl]
    Let $F: D \subseteq \R^3 \rightarrow \R^3$ be a differentiable vector field.
    The vector field $\curl F:  D \subseteq \R^3 \rightarrow \R^3$ is defined
    \begin{align*}
        \curl F & = \nabla \times F                                                                                             \\
                & = \left(\frac{\partial}{\partial x}, \frac{\partial}{\partial y}, \frac{\partial}{\partial z}\right) \times F \\
                & = \left(\frac{\partial F_3}{\partial y} - \frac{\partial F_2}{\partial z},
        \frac{\partial F_1}{\partial z} - \frac{\partial F_3}{\partial x},
        \frac{\partial F_2}{\partial x} - \frac{\partial F_1}{\partial y} \right).
    \end{align*}
\end{definition}
\begin{definition}[Irrotational field]
    A vector field $F: D \subseteq \R^3 \rightarrow \R^3$ such that
    $(\curl F)(\vec{r}) = \vec{0}$ for all $\vec{r} \in D$ is said to be irrotational.
\end{definition}
\begin{theorem}
    Let $f: D \subseteq \R^3 \rightarrow \R^3$ have only continuous
    second-order partial derivatives. Then $\curl (\grad f) = \vec{0}$ (irrotational).
\end{theorem}

\begin{definition}
    Let $F: D \subseteq \R^m \rightarrow \R^m$ be a differentiable vector field.
    We define the divergence of $F$ to be the scalar field $\divv F: D \subseteq \R \rightarrow \R$
    given by
    \begin{equation*}
        (\divv F)(\vec{p}) = \nabla \cdot F = \frac{\partial F_1}{\partial x_1} + \cdots + \frac{\partial F_m}{\partial x_m}.
    \end{equation*}
\end{definition}

\begin{definition}[Laplace operator]
    For scalar fields
    \begin{equation*}
        \nabla^2 f = \frac{\partial^2 f}{\partial x_1^2} + \cdots + \frac{\partial^2 f}{\partial x_m^2}.
    \end{equation*}
    For vector fields
    \begin{equation*}
        \nabla^2 F = (\nabla^2 F_1, \ldots, \nabla^2 F_m).
    \end{equation*}
    Note that $\nabla^2 f = \nabla \cdot \nabla.$
\end{definition}

\begin{notes}[Del operator]
    Let $f, g$ be scalar fields and $F, G$ be vector fields and $c \in \R$.
    \begin{enumerate}
        \item $(\grad f) = \nabla f$
        \item $(\curl F) = \nabla \times F$
        \item $(\divv F) = \nabla \cdot F$
    \end{enumerate}
\end{notes}

\begin{theorem}[Del formulas]
    \begin{enumerate}
        \item $\nabla (f + g) = \nabla f + \nabla g$
        \item $\nabla (cf) = c \nabla f$
        \item $\nabla \times (F + G) = \nabla \times F + \nabla \times$
        \item $\nabla \times (cF) = c\nabla \times F$
        \item $\nabla \cdot (F + G) = \nabla \cdot F + \nabla \cdot G$
        \item $\nabla \cdot (cF) = c(\nabla \cdot F)$
        \item $\nabla (fg) = \nabla fg + f\nabla g $
        \item $\nabla \times (fF) = f\nabla \times F + \nabla f \times F$
        \item $\nabla \cdot (F \times G) = (\nabla \times F) \cdot G - (\nabla \times G) \cdot F$
        \item $\nabla \times (\nabla f) = 0$
        \item $\nabla \cdot (\nabla \times F) = 0$
        \item $\nabla \times (\nabla \times F) = \nabla(\nabla \cdot F) - \nabla^2 F.$
    \end{enumerate}
\end{theorem}

\section{Calculus}
\subsection{Single variable}
\begin{definition}[Limit at infinity of a sequence]
    The sequence $(a_k)$ of real numbers converges to $a \in \R$
    if for each $\epsilon > 0$ there corresponds $K \in \mathbb{N}$ such that
    \begin{equation*}
        \abs{a_k - a} < \epsilon \ \text{whenever} \ k > K.
    \end{equation*}
\end{definition}
\begin{definition}[Cauchy sequence]
    The sequence $(a_k)$ of real numbers is a Cauchy sequence if to each
    $\epsilon > 0$ there corresponds $K \in \R$ such that
    \begin{equation*}
        \abs{a_k - a_l} < \epsilon \ \text{whenever} \ k, l > K.
    \end{equation*}
\end{definition}
\begin{definition}[Continuous function]
    A function $f: D \subseteq \R \rightarrow \R$ is continuous at some point $c$ if
    \begin{equation*}
        \lim_{x \rightarrow c} f(x) = f(c).
    \end{equation*}
    If this is true for all $c \in D$, then we call the function continuous.
\end{definition}
\begin{theorem}[Convergence of a Cauchy sequence]
    A Cauchy sequence is convergent.
\end{theorem}
\begin{theorem}[Bolzano-Weierstrass]
    Every sequence of real numbers in a compact interval $[a, b]$ (a bounded sequence)
    has a subsequence that converges to a limit in $[a, b]$.
\end{theorem}
\begin{theorem}[Intermediate-value theorem]
    Let $f: [a, b] \subseteq \R \rightarrow \R$ be a continuous function. If $t$ is any point between the points $f(a)$ and $f(b)$
    then there exists a point $x \in [a, b]$ such that $f(x) = t$.
\end{theorem}
\begin{theorem}[Mean-value theorem]
    Let $p$ and $h > 0$ be real numbers, and suppose that $f: [p, p + h] \rightarrow \R$
    is a real-valued function such that (a) $f$ is continuous on $[p, p + h]$, and (b) $f$ is differentiable
    at every point of the open interval $(p, p + h)$. Then there exists a point $c$ in the open interval
    $(p, p + h)$ such that
    \begin{equation*}
        f(p + h) = f(p) + hf'(c).
    \end{equation*}
\end{theorem}
\begin{theorem}[Taylor's theorem]
    Let $f: D \subseteq \R \rightarrow \R$ be a real-valued function on an open interval $D$
    in $\R$, and suppose that $f$ is $n$ times differentiable at every point of $D$. If $p$ and $p + h$ are
    points of $D$, then there exists in $D$ a point $c = p + \theta h,\ 0 < \theta < 1$, such that
    \begin{equation*}
        f(p + h) = f(p) + hf'(p) + \frac{h^2}{2!}f''(p) + \cdots + \frac{h^{n-1}}{(n-1)!}f^{(n-1)}(p) + \frac{h^n}{n!}f^{(n)}(c).
    \end{equation*}
\end{theorem}
\begin{theorem}[Integral mean-value theorem]
    If $f: [a, b] \rightarrow \R$ is continuous then there exists $\xi \in (a, b)$ such that
    \begin{equation*}
        \int_{a}^{b} f(x) \, dx\ = (b-a)f(\xi)
    \end{equation*}
\end{theorem}
\begin{definition}[Uniformly continuous function]
    A function $f: D \subseteq \R \rightarrow \R$ is uniformly continuous on $D$ if to each $\epsilon > 0$
    there corresponds $\delta > 0$ (depending only on $\epsilon$) such that
    \begin{equation*}
        \abs{f(t) - f(s)} < \epsilon \quad \textnormal{whenever} \quad \abs{t - s} < \delta, \quad t,s \in D.
    \end{equation*}
\end{definition}

\begin{theorem}
    A function $f: [a, b] \subseteq \R \rightarrow \R$ which is continuous on
    $[a, b]$ is uniformly continuous on $[a, b]$.
\end{theorem}

\begin{theorem}
    Let $g: [c, d] \subseteq \R \rightarrow \R$ be a bounded function and suppose that
    $m \leq g(y) \leq M$ for all $y \in [c, d]$ for all $y \in [c, d]$. If $g$ is integrable over
    $[c, d]$, then
    \begin{equation*}
        m(d - c) \leq \int_{c}^{d} g(y) \, dx\ \leq M(d - c).
    \end{equation*}
\end{theorem}

\begin{definition}[Riemann sum]
    Let ${f: [a, b] \subseteq \R \rightarrow \R}$ be a bounded function
    and let ${P = (x_0, x_1, \ldots, x_n)}$ be a partition of $[a, b]$ such that
    ${a = x_0 < x_1 < \cdots < x_n = b}$. The Riemann sum of $f$ and $P$ is defined
    as
    \begin{equation*}
        R(f, P) = \sum_{i=1}^k f(p_i)(x_i - x_{i-1})
    \end{equation*}
    where $p_i$ is any point in $[x_{i-1}, x_i]$.

    The lower and upper Darboux sums are two special cases of the Riemann sum.
    The lower Darboux sum $L(f, P)$ is defined
    \begin{equation*}
        L(f, P) = \sum_{i=1}^k \inf_{t \in [x_{i-1}, x_i]} f(t)(x_i - x_{i-1})
    \end{equation*}
    and the upper Darboux sum $U(f, P)$ is defined
    \begin{equation*}
        U(f, P) = \sum_{i=1}^k \sup_{t \in [x_{i-1}, x_i]} f(t)(x_i - x_{i-1}).
    \end{equation*}

    In addition, we denote
    \begin{equation*}
        L(f) = \sup_{P} L(f, P)
    \end{equation*}
    and
    \begin{equation*}
        U(f) = \inf_{P} U(f, P).
    \end{equation*}
\end{definition}

\begin{theorem}
    Let ${f: [a, b] \subseteq \R \rightarrow \R}$ be a bounded function.
    For any partitions $P$ and $Q$ of $[a, b]$, it holds that
    $L(f, P) \leq U(f, Q)$.
\end{theorem}

\begin{corollary}
    For all bounded functions ${f: [a, b] \subseteq \R \rightarrow \R}$
    and any partitions $P$ and $Q$ it holds that
    \begin{equation*}
        L(f, P) \leq L(f) \leq U(f) \leq U(f, Q).
    \end{equation*}
\end{corollary}

\begin{definition}[Riemann integrable function]
    The function ${f: [a, b] \subseteq \R \rightarrow \R}$ is said to be
    Riemann integrable over $[a, b]$ if $L(f) = U(f)$.
\end{definition}

\begin{theorem}
    The function ${f: [a, b] \subseteq \R \rightarrow \R}$ is integrable over
    $[a, b]$ if and only if to each $\epsilon > 0$ there corresponds
    partitions $P$ and $Q$ such that
    \begin{equation*}
        U(f, P) - L(f, Q) < \epsilon.
    \end{equation*}
\end{theorem}


\subsection{Multivariate}
\begin{definition}[Limit at infinity of a sequence]
    A sequence $(\vec{a}_k)$ in $\R^n$ is said to converge to $\vec{a} \in \R^n$
    if $\lim_{k \rightarrow \infty} \norm{\vec{a}_k - \vec{a}} = 0$ in $\R$.
    Equivalently, the sequence $(\vec{a}_k)$ converges to $\vec{a}$ if and only if to
    each $\epsilon > 0$ there corresponds a natural number $K$ such that
    \begin{equation*}
        \norm{\vec{a}_k - \vec{a}} < \epsilon \quad \textnormal{whenever} \quad k > K.
    \end{equation*}
\end{definition}

\begin{theorem}
    A sequence in $\R^n$, $(\vec{a}_k) = (a_{k1}, \ldots, a_{kn})$, converges to $\vec{a} = (a_1, \ldots, a_n)$ if and only if
    $a_{ki} \rightarrow a_i$ for each $i = 1,\ldots,n$.
\end{theorem}

\begin{definition}[Continuous function]
    The function $f: S \subseteq \R^m \rightarrow \R^n$
    is continuous at $\vec{p} \in S$ if, for any sequence
    $(\vec{a}_k)$ in $S$, $f(\vec{a}_k) \rightarrow \vec{p}$ whenever
    $(\vec{a}_k) \rightarrow \vec{p}$. The function $f$ is continuous
    if and only if it is continuous at every point of $S$.
\end{definition}

\begin{theorem}
    The function $f: S \subseteq \R^m \rightarrow \R^n$
    is continuous at $\vec{p}$ if and only if for each $i=1, \ldots, n$ the coordinate function
    $f_i: S \subseteq \R^m \rightarrow \R$ is continuous at $\vec{p}$.
\end{theorem}

\begin{corollary}
    Let $D$ be an open subset of $\R^m$. A function $f: D \subseteq \R^m \rightarrow \R$
    is continuously differentiable if and only if $\nabla f$ is continuous.
\end{corollary}

\begin{corollary}
    Any linear function $f: \R^m \rightarrow \R^n$ is continuous.
\end{corollary}

\begin{definition}[Limit]
    Given a cluster point $\vec{p}$ of $S \subseteq \R^m$, a point $\vec{q}$ of $\R^n$,
    and a function ${f: S \subseteq \R^m \rightarrow \R^n}$, we write
    ${\lim_{\vec{x} \rightarrow \vec{p}} f(\vec{x})} = \vec{q}$
    if $f(\vec{a}_k) \rightarrow \vec{q}$ whenever $\vec{a}_k \rightarrow \vec{p}$ where the sequence
    $(\vec{a}_k)$ lies in $S$ and $\vec{a}_k \neq \vec{p}$ for all $k \in \mathbb{N}$.
\end{definition}

\begin{theorem}
    Given a function ${f: S \subseteq \R^m \rightarrow \R^n}$ and $\vec{p}$, a cluster point of $S$,
    then $\lim_{\vec{x} \rightarrow \vec{p}} f(\vec{x}) = \vec{q}$ if and only if to each $\epsilon > 0$ there corresponds
    $\delta > 0$ such that $\norm{f(\vec{x}) - \vec{q}} < \epsilon$ whenever $\vec{x} \in S$ and
    $0 < \norm{\vec{x} - \vec{p}} < \delta$.
\end{theorem}

\begin{theorem}
    The function ${f: S \subseteq \R^m \rightarrow \R^n}$ is continuous at $\vec{p} \in S$
    if and only if  $\lim_{\vec{x} \rightarrow \vec{p}} f(\vec{x}) = f(\vec{p})$.
\end{theorem}

\begin{theorem}
    The function ${f: S \subseteq \R^m \rightarrow \R^n}$ is continuous at $\vec{p} \in S$
    if and only if to each $\epsilon > 0$ there corresponds $\delta > 0$ such that $f(\vec{x}) \in N(f(\vec{p}), \epsilon)$
    whenever $\vec{x} \in N(\vec{p}, \delta)$.
\end{theorem}

\begin{theorem}
    A function ${f: D \subseteq \R^m \rightarrow \R^n}$, where $D$ is an
    open subset of $\R^m$, is continuous if and only if for each open subset $V$
    of $\R^n$ the inverse image
    \begin{equation*}
        f^{-1}(V) = \{\vec{x} \in \R^m : \vec{x} \in D, f(\vec{x}) \in V\}
    \end{equation*}
    is open in $\R^m$.
\end{theorem}

\begin{theorem}
    A function ${f: S \subseteq \R^m \rightarrow \R^n}$ is uniformly continuous
    on $S$ if to each $\epsilon > 0$ there corresponds $\delta > 0$ such that, for all
    $\vec{x}, \vec{y} \in S$,
    \begin{equation*}
        \norm{f(\vec{x}) - f(\vec{y})} < \epsilon \quad \textnormal{whenever} \quad \norm{\vec{x} - \vec{y}} < \delta.
    \end{equation*}
\end{theorem}

\begin{theorem}
    Let $f: K \subseteq \R^m \rightarrow \R^n$ be a continuous function
    on a compact domain $K$. Then
    \begin{enumerate}
        \item $f$ is uniformly continuous on $K$.
        \item The image $f(K)$ is compact in $\R^n$.
    \end{enumerate}
\end{theorem}

\begin{theorem}
    Let $f: K \subseteq \R^m \rightarrow \R^n$ be a 1-1 continuous
    function defined on a compact domain $K$. Then the inverse function
    $f^{-1}: f(K) \subseteq \R^n \rightarrow \R^m$ is continuous.
\end{theorem}

\begin{definition}
    Given a function $f: D \subseteq \R^m \rightarrow \R^n$,
    where $D$ is an open subset of $\R^m$, and a point $\vec{p} \in D$,
    the difference function $\delta_{f, \vec{p}} : D_{\vec{p}} \subseteq \R^m \rightarrow \R^n$
    is defined by
    \begin{equation*}
        \delta_{f, \vec{p}}(\vec{h}) = f(\vec{p} + \vec{h}) - f(\vec{p}).
    \end{equation*}
\end{definition}

\begin{definition}
    Let $g: N \subseteq \R^m \rightarrow \R^n$ and $g^*: M \subseteq \R^m \rightarrow \R^n$ be
    two functions whose open domains both contain $\vec{0}$. We will say that $g$ and $g*$
    closely approximate each other near $\vec{0}$ if there is a function
    $\eta: N \cap M \subseteq \R^m \rightarrow \R^n$ such that
    \begin{enumerate}
        \item $g(\vec{h}) - g^*(\vec{h}) = \norm{\vec{h}}\eta(\vec{h})$
        \item $\eta(\vec{h}) \rightarrow \vec{0}$ when $\vec{h} \rightarrow \vec{0}$
    \end{enumerate}
\end{definition}

\begin{definition}
    A function $f: D \subseteq \R^m \rightarrow \R^n$, defined on an open subset $D$
    of $\R^m$, is differentiable at $\vec{p} \in D$ if the difference function
    $\delta_{f, \vec{p}} : D_{\vec{p}} \subseteq \R^m \rightarrow \R^n$ can be closely
    approximated by a linear function $L: \R^m \rightarrow \R^n$ near $\vec{0}$. The function
    is said to be differentiable if it is differentiable everywhere in $D$.
\end{definition}

\begin{theorem}
    A function $f: D \subseteq \R^m \rightarrow \R^n$ is differentiable
    at $\vec{p}$ if and only if there is a linear function $L: \R^m \rightarrow \R^n$
    and a function $\eta D_{\vec{p}} \subseteq \R^m \rightarrow \R^n$ such that
    \begin{equation*}
        f(\vec{p} + \vec{h}) - f(\vec{p}) = L(\vec{h}) + \norm{\vec{h}}\eta{\vec{h}}.
    \end{equation*}
    and
    \begin{equation*}
        \lim_{\vec{h} \rightarrow \vec{0}} \eta(\vec{h}) = \vec{0}
    \end{equation*}
\end{theorem}
\subsection{Line integrals}
\begin{definition}[Path (line) integral]
    For some scalar field $f: U \subseteq \R^n \rightarrow \R$,
    the path integral along a (piecewise) $C^1$ path $\alpha: [a, b] \subseteq \R \rightarrow \R^n$
    whose image is contained in $U$ is defined as
    \begin{equation*}
        \int_{C} f \, ds\ =\int_{a}^{b} f(\alpha(t))\norm{\alpha'(t)} \, dt\
    \end{equation*}
\end{definition}

\section{Geometry}
\begin{definition}[Parametrization]
    Let $f: S \subseteq \R^m \rightarrow \R^n$. $f$ is a
    parametrization of a curve, a surface, or more generally, a manifold $M$
    defined by an implicit equation, if its coordinate functions are the parametric equations
    of $M$.
\end{definition}

\section{Topology}
\begin{definition}[Cluster point]
    Let $S$ be a subset of $\R^m$. A point $\vec{c}$ is
    called a cluster point (or a limit point) of $S$ if there is a
    sequence of points $(\vec{x}_k)$ in $S$ such that $\vec{x}_k \neq \vec{c}$ for
    all $k \in \mathbb{N}$, but $\vec{x}_k \rightarrow \vec{c}$.
\end{definition}

\begin{definition}[Neighbourhood of a point]
    If $X$ is a topological space and $p$ is a point in $X$,
    then a neighbourhood of $p$ is a subset $V$ of $X$ that
    includes an open set $U$ containing $p$,
    \begin{equation*}
        p \in U \subseteq V \subseteq X.
    \end{equation*}
    This is equivalent to the point $p \in X$ beloning to the
    interior of $V$ in $X$.
\end{definition}

\begin{definition}[Bounded subset]
    A subset $S \subseteq \R^m$ is bounded if it is contained in a closed
    ball $B(\vec{p} \in \R^m, k), k > 0$.
\end{definition}

\begin{definition}[Compact subset]
    A subset $S \subseteq \R^m$ is compact if it is closed and bounded in $\R^m$.
\end{definition}

\begin{definition}[Curve]
    A subset $C$ of $\R^n$ is a curve if there is a $C^1$ function
    $f: D \subseteq \R \rightarrow \R^n$, where $D$ is an interval, such that
    $f(D) = C$. The function $f$ is called a $C^1$ parametrization of the curve.
\end{definition}

\begin{definition}[Smooth function]
    A function $f: D \subseteq \R \rightarrow \R^n$ is said to be smooth if it is
    $C^1$ and if $f'(t) \neq \vec{0}$ for all $t \in D$.
\end{definition}

\begin{definition}[Simple arc]
    A curve $C$ in $\R^n$ is a (smooth) simple arc if $C$ has a 1-1
    (smooth) $C^1$ parametrization of the form $f: [a, b] \subseteq \R \rightarrow \R^n$.
    The points $f(a)$ and $f(b)$ are then called the end points of the arc.
    The function $f$ is called a simple parametrization of $C$.
\end{definition}

\begin{definition}[Path]
    A continuous function $f: [a, b] \in \R \rightarrow \R^n$ is called a
    path in $\R^n$ from $f(a)$ to $f(b)$.
\end{definition}

\section{Linear algebra}

\subsection{Dimensionality}
\begin{theorem}[Rank-nullity]
    Let $L: V \rightarrow W$ be a linear function
    where $V$ is finite dimensional. Then
    \begin{equation}
        \dim \im L + \dim \ker L = \dim V.
    \end{equation}
\end{theorem}
\subsection{Inequalities}


\begin{theorem}[Cauchy-Schwarz inequality]
    For any vectors $\vec{a}, \vec{b}$ in an inner product space
    \begin{equation}
        \inner{\vec{a}, \vec{b}} \leq \norm{\vec{a}} \norm{\vec{b}}.
    \end{equation}
\end{theorem}

\begin{theorem}[Triangle inequality]
    For any vectors $\vec{a}, \vec{b}$ in an inner product space
    \begin{equation}
        \norm{\vec{a} + \vec{b}} \leq \norm{\vec{a}} + \norm{\vec{b}}.
    \end{equation}
\end{theorem}
\begin{corollary}
    \begin{enumerate}
        \item For any vectors $\vec{a}, \vec{b}$ in an inner product space
              \begin{equation}
                  \abs{\norm{ \vec{a}} - \norm{ \vec{b}}} \leq \norm{\vec{a} - \vec{b}} \leq \norm{\vec{a}} + \norm{ \vec{b}}.
              \end{equation}
        \item For any vectors $\vec{a_1}, \ldots, \vec{a_n}$ in an inner product space and scalars $x_1, \ldots, x_n$,
              \begin{equation}
                  \norm{x_1\vec{a_1} + \cdots + x_n\vec{a_n}} \leq |x_1| \norm{\vec{a_1}} + \cdots + \abs{x_n}\norm{\vec{a_n}}.
              \end{equation}
        \item For any $\vec{x} = (x_1, \ldots, x_n)$,
              \begin{equation}
                  \norm{(x_1, \ldots, x_n)} \leq \abs{x_1} + \cdots + \abs{x_n}.
              \end{equation}
    \end{enumerate}
\end{corollary}

\end{document}
