\documentclass{article}
\usepackage{ntheorem, amsmath}
\usepackage{graphicx} % Required for inserting images
\usepackage[framemethod=tikz]{mdframed}

\let\vec\mathbf{}

\theoremstyle{break}
\newmdtheoremenv[
  linecolor=red,
  roundcorner=5pt,
]{definition}{Definition}

\theoremstyle{break}
\newmdtheoremenv[
  roundcorner=5pt,
  hidealllines=true,
  leftline=true,
  linecolor=gray
]{theorem}{Theorem}


\theoremstyle{break}
\newmdtheoremenv[
  roundcorner=5pt,
  hidealllines=true,
  leftline=true,
  linecolor=gray
]{corollary}{Corollary}[theorem]
\let\vec\mathbf{}
\let\t\textnormal{}
\DeclareMathOperator{\im}{im}
\DeclareMathOperator{\curl}{curl}
\DeclarePairedDelimiter{\norm}{\lVert}{\rVert}
\DeclarePairedDelimiter{\inner}{\langle}{\rangle}
\DeclarePairedDelimiter{\abs}{\lvert}{\rvert}

\title{Science notes 2}
\author{Juho Lamminmäki }
\date{March 2024}

\begin{document}

\maketitle

\section{Functions}
\subsection{Vector-valued functions}
\begin{definition}[Vector-valued function]
    A real vector-valued function $f: S \subseteq \mathbb{R}^m \rightarrow \mathbb{R}^n$
    is a rule that assigns to each $\vec{p} \in S$ a unique $f(\vec{p}) \in \mathbb{R}^n$.
\end{definition}
\begin{definition}[Coordinate function]
    A vector-valued function $f: S \subseteq \mathbb{R}^m \rightarrow \mathbb{R}^n$ may be expressed
    as
    \begin{equation}
        f(\vec{p}) = (f_1(\vec{p}), \ldots, f_n(\vec{p}))
    \end{equation}
    where $f_i: S \subseteq \mathbb{R}^m \rightarrow \mathbb{R}$ are called coordinate functions of $f$.
\end{definition}
\begin{definition}[Curl]
    Let $F: D \subseteq \mathbb{R}^3 \rightarrow \mathbb{R}^3$ be a differentiable vector field.
    The vector field $\curl F:  D \subseteq \mathbb{R}^3 \rightarrow \mathbb{R}^3$ is defined
    \begin{align}
        \curl F & = \Delta \times F                                                                                             \\
                & = \left(\frac{\partial}{\partial x}, \frac{\partial}{\partial y}, \frac{\partial}{\partial z}\right) \times F \\
                & = \left(\frac{\partial F_3}{\partial y} - \frac{\partial F_2}{\partial z},
        \frac{\partial F_1}{\partial z} - \frac{\partial F_3}{\partial x},
        \frac{\partial F_2}{\partial x} - \frac{\partial F_1}{\partial y} \right).
    \end{align}
\end{definition}

\section{Calculus}
\begin{definition}[Limit at infinity of a sequence]
    The sequence $(a_k)$ of real numbers converges to $a \in \mathbb{R}$
    if for each $\epsilon > 0$ there corresponds $K \in \mathbb{N}$ such that
    \begin{math}
        \abs{a_k - a} < \epsilon \ \text{whenever} \ k > K.
    \end{math}
\end{definition}
\begin{definition}[Cauchy sequence]
    The sequence $(a_k)$ of real numbers is a Cauchy sequence if to each
    $\epsilon > 0$ there corresponds $K \in \mathbb{R}$ such that
    \begin{math}
        \abs{a_k - a_l} < \epsilon \ \text{whenever} \ k, l > K.
    \end{math}
\end{definition}
\begin{theorem}[Convergence of a Cauchy sequence]
    A Cauchy sequence is convergent.
\end{theorem}
\begin{theorem}[Bolzano-Weierstrass]
    Every sequence of real numbers in a compact interval $[a, b]$ (a bounded sequence)
    has a subsequence that converges to a limit in $[a, b]$.
\end{theorem}
\begin{theorem}[Intermediate-value theorem]
    Let $f: [a, b] \subseteq \mathbb{R} \rightarrow \mathbb{R}$ be a continuous function. If $t$ is any point between the points $f(a)$ and $f(b)$
    then there exists a point $x \in [a, b]$ such that $f(x) = t$.
\end{theorem}

\section{Geometry}
\begin{definition}[Parametrization]
    Let $f: S \subseteq \mathbb{R}^m \rightarrow \mathbb{R}^n$. $f$ is a
    parametrization of a curve, a surface, or more generally, a manifold $M$
    defined by an implicit equation, if its coordinate functions are the parametric equations
    of $M$.
\end{definition}

\section{Linear algebra}

\subsection{Dimensionality}
\begin{theorem}[Rank-nullity]
    Let $L: V \rightarrow W$ be a linear function
    where $V$ is finite dimensional. Then
    \begin{equation}
        \dim \im L + \dim \ker L = \dim V.
    \end{equation}
\end{theorem}
\subsection{Inequalities}


\begin{theorem}[Cauchy-Schwarz inequality]
    For any vectors $\vec{a}, \vec{b}$ in an inner product space
    \begin{equation}
        \inner{\vec{a}, \vec{b}} \leq \norm{\vec{a}} \norm{\vec{b}}.
    \end{equation}
\end{theorem}

\begin{theorem}[Triangle inequality]
    For any vectors $\vec{a}, \vec{b}$ in an inner product space
    \begin{equation}
        \norm{\vec{a} + \vec{b}} \leq \norm{\vec{a}} + \norm{\vec{b}}.
    \end{equation}
\end{theorem}
\begin{corollary}
    \begin{enumerate}
        \item For any vectors $\vec{a}, \vec{b}$ in an inner product space
              \begin{equation}
                  \abs{\norm{ \vec{a}} - \norm{ \vec{b}}} \leq \norm{\vec{a} - \vec{b}} \leq \norm{\vec{a}} + \norm{ \vec{b}}.
              \end{equation}
        \item For any vectors $\vec{a_1}, \ldots, \vec{a_n}$ in an inner product space and scalars $x_1, \ldots, x_n$,
              \begin{equation}
                  \norm{x_1\vec{a_1} + \cdots + x_n\vec{a_n}} \leq |x_1| \norm{\vec{a_1}} + \cdots + \abs{x_n}\norm{\vec{a_n}}.
              \end{equation}
        \item For any $\vec{x} = (x_1, \ldots, x_n)$,
              \begin{equation}
                  \norm{(x_1, \ldots, x_n)} \leq \abs{x_1} + \cdots + \abs{x_n}.
              \end{equation}
    \end{enumerate}
\end{corollary}

\end{document}
